\subsection{Бифуркация Андронова--Хопфа}
\begin{definition}
        Замкнутую траеторию $\varphi(t)$ системы \ref{eq:common_continuous_system} будем называть предельным циклом, если в окрестности этой траектории нет других замкнутых орбит.
        \cite[стр.~182]{bratus10}
\end{definition}
\begin{definition}
        Бифукрация положения равновесия, соответствующая появлению сопряженных чисто мнимых собственных чисел, называется бифуркацией Пуанкаре--Андронова--Хопфа или бифуркацией рождения цикла.
        \cite[стр.~192]{bratus10}
\end{definition}

Как было замечено ранее, в изучаемой системе \ref{eq:short_continuous_system} все собственные значения матрицы Якоби действительны и в ней не могут появиться чисто мнимые собственные значения ни при каких значениях параметров. Это говорит о невозможности возникновения бифуркации Андронова--Хопфа.
