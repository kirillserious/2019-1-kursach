\subsection{Формулировка задачи}

Исследовать нелинейную динамику следующей системы на плоскости:
\begin{equation}\label{eq:continuous_system}
        \left\{
        \begin{array}{ll}
                \dot x = rx(b - \ln x) - bxy \\
                \dot y = -cy + \frac{dxy}{N+y}
        \end{array}
        \right.,
\end{equation}
где $b,\; c,\; d,\; N$~--- положительные параметры и $(x, y) \in \R^2_+$. Необходимо:
\begin{enumerate}
        \item 
                дать биологическую интерпретацию характеристик системы;
        \item
                ввести новые безразмерные переменные, максимально уменьшив число входящих параметров, выбрать два свободных параметра;
        \item
                найти неподвижные точки системы и исследовать их характер в зависимости от значений параметров, представить результаты исследования в виде параметрического портрета системы;
        \item
                для каждой характерной области параметрического портрета построить фазовый портрет, дать характеристику поведения системы в каждом из этих случаев;
        \item
                исследовать возможность возникновения предельного цикла. В положительном случае найти соответствующее первое ляпуновское число, исследовать характер предельного цикла (устойчивый, неустойчивый, полуустойчивый);
        \item
                дать биологическую интерпретацию полученным результатам.
\end{enumerate}