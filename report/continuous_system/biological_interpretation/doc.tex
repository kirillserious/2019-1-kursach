\subsection{Биологическая интерпретация системы}

Система \ref{eq:continuous_system} подходит под описание модели <<хищник-жертва>>, которую в общем случае можно описать как
$$
        \left\{
        \begin{array}{ll}
                \dot u = A(u) -B(u, v) \\
                \dot v = -C(v) +D(u, v),
        \end{array}
        \right.
$$
где\\
$u, v$ --- численность жертв и хищников соответственно,\\
$A(u)$ --- функция, описывающая размножение жертв в отсутствии хищников, \\
$C(v)$ --- функция, описывающая вымирание хищников при отсутствии жертв, \\
$B(u, v)$ --- выедание жертв хищниками, \\
$D(u, v)$ --- эффективность поедания жертв хищниками. \cite[стр.~137]{bratus10}

В случае системы \ref{eq:continuous_system} в функцию $A(x) = rx(b - \ln x)$ введен член, ограничивающий рост популяции жертв, максимально возможное число которых задаётся параметром $b$ и составляет $e^b$. Такая модель учитывает внутривидовую конкуренцию среди жертв.

Функция $C(y) = cy$ --- линейная. Это говорит о том, что хищники в рамках модели не конкурируют за ресурсы, отличные от жертв (например, за территорию). Такая модель может описывать взаимодействие популяций при небольшом количестве хищников. Параметр отвечает за продолжительность жизни хищников: чем он больше, тем быстрее хищники умирают.

Рассмотрим функцию $B(x, y) = bxy$. Для интерпретации системы можно ввести \textit{трофическую функцию} $B(x, \cdot)$, которая показывает зависимость выедания жертвы при фиксированном числе хищников. В нашем случае трофическая функция линейна, что говорит об отсутствии насыщения хищников, а также отсутствии конкуренции за добычу.
