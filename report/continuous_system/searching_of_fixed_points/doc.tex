\subsection{Поиск неподвижных точек}
\begin{definition}
        Точка $u_0 \in U \subseteq \R^n$ называется неподвижной точкой динамической системы $\dot u = f(u)$, $f: U \rightarrow \R^n$, если $f(u_0) = 0$. \cite[стр.~20]{bratus10}
\end{definition}

Найдем неподвижные точки итоговой системы \ref{eq:short_continuous_system}. Для этого решим следующую систему:
$$
        \left\{
        \begin{aligned}
                u(\tau)\left( -\alpha\ln u(\tau) - \beta v(\tau) \right) &= 0 \\
                v(\tau)\left( -1 + \frac{u(\tau)}{\gamma + v(\tau)}\right) &= 0.
        \end{aligned}
        \right.
$$
Исходя из вида полученной системы уравнений легко обнаружить неподвижные точки $A (0,\;0)$ и $B (1,\; 0)$. Найдем в общем случае остальные неподвижные точки, исходя из предположения, что $u(\tau) \neq 0$ и $v(\tau) \neq 0$:
$$
        \left\{
        \begin{aligned}
                -\alpha\ln u(\tau) - \beta v(\tau)  &= 0 \\
                -1 + \frac{u(\tau)}{\gamma + v(\tau)} &= 0.
        \end{aligned}
        \right.
        \;
        \Longrightarrow
        \;
        \left\{
        \begin{aligned}
                u(\tau) &= e^{-\frac{\beta}{\alpha}v(\tau)} \\
                u(\tau) &= v(\tau) + \gamma.
        \end{aligned}
        \right.  
$$
Полученная система не решается аналитически для произвольных положительных значениях параметров $\alpha,\; \beta$ и $\gamma$. В рамках задачи разрешено фиксировать значения параметров, если их больше двух. Зафиксируем значение параметра $\gamma = 1$. В таком случае у последней системы не будет существовать решений, и дальнейшие рассуждения будут приведены для неподвижных точек $A$ и $B$.